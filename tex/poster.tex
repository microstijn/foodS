\documentclass[a4paper, 10pt]{article}

% PACKAGES
\usepackage[utf8]{inputenc}
\usepackage[T1]{fontenc}
\usepackage{lmodern} % Modern font
\usepackage[dutch]{babel}
\usepackage[margin=1.2cm, top=1.5cm, bottom=2.5cm]{geometry} % Narrow margins, space for footer
\usepackage{graphicx} % For including images
\usepackage{multicol} % For the three-column layout
\usepackage[dvipsnames, svgnames]{xcolor} % For custom colors
\usepackage{tcolorbox} % For colored boxes
\usepackage{fontawesome5} % For icons (requires lualatex or xelatex)
\usepackage{fancyhdr} % For the footer
\usepackage{tikz} % For drawing footer background and placeholders

% --- CUSTOMIZATION ---
% Define colors for consistency
\definecolor{headerRed}{RGB}{217, 83, 79}
\definecolor{headerOrange}{RGB}{240, 173, 78}
\definecolor{headerBlue}{RGB}{91, 192, 222}
\definecolor{footerGray}{RGB}{245, 245, 245}

% Tcolorbox style for column headers
\newtcolorbox{columnheader}[2][]{
	coltext=black,
	fontupper=\large\bfseries,
	boxrule=1pt,
	arc=2mm,
	halign=center,
	valign=center
}

% --- NEW: Command for image placeholders ---
\newcommand{\placeholderimage}[3]{%
	\begin{tikzpicture}
		\draw[gray, thick, rounded corners=3pt] (0,0) rectangle (#1, #2);
		\node[text width=0.9*#1, align=center, text=gray] at (#1/2, #2/2) {\small\bfseries #3};
	\end{tikzpicture}%
}


% --- DOCUMENT SETUP ---
\pagestyle{empty} % No page numbers
\setlength{\columnsep}{1cm} % Space between columns
\setlength{\parindent}{0pt} % No indentation

% --- FOOTER SETUP ---
\pagestyle{fancy}
\fancyhf{} % Clear all header and footer fields
\renewcommand{\headrulewidth}{0pt} % No header rule
\fancyfoot[C]{
	\begin{tikzpicture}[remember picture, overlay]
		\node[yshift=1.25cm] at (current page.south) {
			\begin{tikzpicture}[remember picture, overlay]
				% Footer Background Box
				\fill[footerGray] (current page.south west) rectangle (current page.south east);
				
				% Left Side: Contact Info
				\node[anchor=west, text width=0.45\textwidth, align=left, inner sep=10pt] at ([xshift=1.2cm]current page.south west) {
					\large\textbf{Dode dieren melden?} \\
					Bel Waterschap Rivierenland: \textbf{(0344) 64 90 90} \\[.5em]
					\small\textit{Disclaimer: Deze gids is een hulpmiddel. De officiële GLOBALG.A.P. regelgeving is leidend.}
				};
				
				% Right Side: QR Code
				\node[anchor=east, text width=0.45\textwidth, align=right, inner sep=10pt] at ([xshift=-1.2cm]current page.south east) {
					\begin{minipage}{0.2\textwidth}
						\centering
						\placeholderimage{\linewidth}{\linewidth}{QR Code}
					\end{minipage}%
					\begin{minipage}{0.7\textwidth}
						\large\textbf{Meer Weten? Scan de QR-code} \\
						\texttt{uw-github-paginanaam.github.io}
					\end{minipage}%
				};
			\end{tikzpicture}
		};
	\end{tikzpicture}
}


% --- DOCUMENT START ---
\begin{document}
	
	% --- HEADER ---
	\begin{minipage}[t]{0.75\textwidth}
		\Huge\bfseries Handvat Irrigatiewater: Veilig \& Simpel \\[0.2em]
		\large\textit{Praktische Gids voor GLOBALG.A.P.-conform watergebruik}
	\end{{minipage}%
	\begin{minipage}[t]{0.25\textwidth}
		\raggedleft
		\placeholderimage{3.5cm}{1.5cm}{Logo Waterschap}
	\end{minipage}%
	\vspace{0.5cm}
	\hrule
	\vspace{0.5cm}	
	% --- THREE-COLUMN LAYOUT START ---
	\begin{multicols}{3}		
		% --- COLUMN 1: DIRECT DOEN ---
		\begin{columnheader}{headerRed}
			Wat als ik NU water nodig heb?
		\end{columnheader}
		\vspace{0.3cm}
		\textbf{Actieplan bij Acute Waterbehoefte} \\
		\placeholderimage{\linewidth}{7cm}{Schema 2: Actieplan}
		\vspace{0.5cm}
		\textbf{Risico Matrix: Hoe lang wachten?} \\
		\placeholderimage{\linewidth}{5cm}{Risico Matrix}
		% --- COLUMN 2: EERST DENKEN ---
		\begin{columnheader}{headerOrange}
			Moet ik wachten?
		\end{columnheader}
		\vspace{0.3cm}
		\textbf{Risico-inventarisatie \& Wachttijd} \\
		\placeholderimage{\linewidth}{7cm}{Schema 1: Risico-inventarisatie}
		\vspace{0.5cm}
		\textbf{Waar moet ik op letten?}
		\begin{itemize}
			\item[\faCloudShowersHeavy] \textbf{Hevige Regen:} Kans op riooloverstort.
			\item[\faPoo] \textbf{Mestafspoeling:} Risico na bemesting.
			\item[\faDove] \textbf{Dieren:} Grote groepen vogels of ratten.
			\item[\faIndustry] \textbf{Lozingen:} Let op pijpen van zuiveringen.
		\end{itemize}	
		% --- COLUMN 3: GOED OM TE WETEN ---
		\begin{columnheader}{headerBlue}
			Achtergrond \& Tips
		\end{columnheader}
		\vspace{0.3cm}
		\textbf{De Boosdoeners (kort)}
		\begin{itemize}
			\item \textbf{Legionella:} Gevaar bij verneveling.
			\item \textbf{Ziekte van Weil:} Risico via rattenurine.
			\item \textbf{Botulisme:} Gevaar door dode dieren.
		\end{itemize}
		\vspace{0.3cm}
		\textbf{Slimme Keuzes ("No-Regret")}
		\begin{itemize}
			\item[\faTint] \textbf{Gebruik druppelirrigatie:} Altijd de veiligste optie.
			\item[\faTint] \textbf{Kies een slim inlaatpunt:} Ver weg van risico's.
			\item[\faSeedling] \textbf{Leg een bufferstrook aan:} Vermindert afspoeling.
		\end{itemize}
		\vspace{0.3cm}
		\textbf{Meten = Weten}
		\begin{itemize}
			\item \textbf{De 3x Regel:} Meet 3x bij een terugkerend risico.
			\item \textbf{Monstername:} Neem monster in het midden (30cm diep).
		\end{itemize}
	\end{multicols}
	% --- THREE-COLUMN LAYOUT END ---
\end{document}
